%iffalse
\let\negmedspace\undefined
\let\negthickspace\undefined
\documentclass[journal,12pt,onecolumn]{IEEEtran}
\usepackage{cite}
\usepackage{amsmath,amssymb,amsfonts,amsthm}
\usepackage{algorithmic}
\usepackage{graphicx}
\usepackage{textcomp}
\usepackage{xcolor}
\usepackage{txfonts}
\usepackage{listings}
\usepackage{enumitem}
\usepackage{mathtools}
\usepackage{gensymb}
\usepackage{comment}
\usepackage{multicol}
\usepackage[breaklinks=true]{hyperref}
\usepackage{tkz-euclide} 
\usepackage{listings}
\usepackage{gvv}                                        
%\def\inputGnumericTable{}                                 
\usepackage[latin1]{inputenc}                                
\usepackage{color}                                            
\usepackage{array}                                            
\usepackage{longtable}                                       
\usepackage{calc}                                             
\usepackage{multirow}                                         
\usepackage{hhline}                                           
\usepackage{ifthen}                                           
\usepackage{lscape}
\usepackage{tabularx}
\usepackage{array}
\usepackage{float}


\newtheorem{theorem}{Theorem}[section]
\newtheorem{problem}{Problem}
\newtheorem{proposition}{Proposition}[section]
\newtheorem{lemma}{Lemma}[section]
\newtheorem{corollary}[theorem]{Corollary}
\newtheorem{example}{Example}[section]
\newtheorem{definition}[problem]{Definition}
\newcommand{\BEQA}{\begin{eqnarray}}
\newcommand{\EEQA}{\end{eqnarray}}
\renewcommand{\define}{\stackrel{\triangle}{=}}
\theoremstyle{remark}
\newtheorem{remark}{Remark}

% Marks the beginning of the document
\begin{document}
\bibliographystyle{IEEEtran}
\vspace{3cm}

\title{MA-$2012$-$14$ to $26$}
\author{AI24BTECH11017 - Maanya}
\maketitle
\bigskip
\renewcommand{\thefigure}{\theenumi}
\renewcommand{\thetable}{\theenumi}
\begin{enumerate}[start=14]
\item Let \( R = \mathbb{Z} \times \mathbb{Z} \) and \( I = \mathbb{Z} \times \mathbb{Z} \setminus \{0\} \).  
Then which of the following statements is correct?

\begin{enumerate}
    \item \( I \) is a maximal ideal but not a prime ideal of \( R \).
    \item \( I \) is a prime ideal but not a maximal ideal of \( R \).
    \item \( I \) is both maximal ideal as well as a prime ideal of \( R \).
    \item \( I \) is neither a maximal ideal nor a prime ideal of \( R \).
\end{enumerate}
\item The function \( u(r, \theta) \) satisfying the Laplace equation
\[
\frac{\partial^2 u}{\partial r^2} + \frac{1}{r} \frac{\partial u}{\partial r} + \frac{1}{r^2} \frac{\partial^2 u}{\partial \theta^2} = 0, \quad e < r < e^2
\]
subject to the conditions \( u(e, \theta) = 1 \), \( u(e^2, \theta) = 0 \) is

\begin{enumerate}
    \item \( \ln(e / r) \)
    \item \( \ln(e / r^2) \)
    \item \( \ln(e^2 / r) \)
    \item \( \sum_{n=1}^{\infty} \left( \frac{r - e^2}{e - e^2} \right) \sin n\theta \)
\end{enumerate}
\item The functional
\[
\int_0^1 \left( y^2 + (y' + 2y')^2 + kyy' + y^2 \right) \, dx, \quad y(0) = 0, \, y(1) = 1, \, y'(0) = 2, \, y'(1) = 3
\]
is path independent if \( k \) equals

\begin{enumerate}
    \item 1
    \item 2
    \item 3
    \item 4
\end{enumerate}
\item If a transformation \( y = u v \) transforms the given differential equation 
\[
f(x)y'' - 4 f'(x) y' + g(x)y = 0 
\]
into the equation of the form \( v'' + h(x)v = 0 \), then \( u \) must be

\begin{enumerate}
    \item \( 1 / f^2 \)
    \item \( x f \)
    \item \( 1 / 2 f \)
    \item \( f^2 \)
\end{enumerate}

\item The expression 
\[
\frac{1}{D_x^2 - D_y^2} \sin(x - y) 
\]
is equal to

\begin{enumerate}
    \item \( -\frac{x}{2} \cos(x - y) \)
    \item \( -\frac{x}{2} \sin(x - y) + \cos(x - y) \)
    \item \( \frac{x}{2} \cos(x - y) + \sin(x - y) \)
    \item \( \frac{3x}{2} \sin(x - y) \)
\end{enumerate}

\item The function \( \phi(x) \) satisfying the integral equation
\[
\int_0^x e^{-t^2} \phi(\xi) \, d\xi = \frac{x^2}{2}
\]
is

\begin{enumerate}
    \item \( \frac{x^2}{2} \)
    \item \( x + \frac{x^2}{2} \)
    \item \( x - \frac{x^2}{2} \)
    \item \( 1 + \frac{x^2}{2} \)
\end{enumerate}

\item Given the data:

\[
\renewcommand{\arraystretch}{1.2}
\setlength{\arrayrulewidth}{0.6pt}
\begin{array}{|c|c|c|c|c|c|}
\hline
x & 1 & 2 & 3 & 4 & 5 \\ 
\hline
y & -1 & 2 & -3 & 4 & -5 \\ 
\hline
\end{array}
\]


If the derivative of \( y(x) \) is approximated as:
\[
y'(x_i) = \frac{1}{h} \left( \Delta y_i + \frac{1}{2} \Delta^2 y_i - \frac{1}{4} \Delta^3 y_i \right),
\]
then the value of \( y'(2) \) is

\begin{enumerate}
    \item 4
    \item 8
    \item 12
    \item 16
\end{enumerate}

\item If 
\[
A = \begin{pmatrix}
1 & 0 & 0 \\
50 & 1 & 0 \\
50 & 0 & 1
\end{pmatrix},
\]
then \( A^{50} \) is

\begin{enumerate}
    \item \( 
    \begin{pmatrix}
    1 & 0 & 0 \\
    50 & 1 & 0 \\
    50 & 0 & 1 
    \end{pmatrix} 
    \)
    \item \( 
    \begin{pmatrix}
    1 & 0 & 0 \\
    48 & 0 & 0 \\
    48 & 0 & 1 
    \end{pmatrix} 
    \)
    \item \( 
    \begin{pmatrix}
    1 & 0 & 0 \\
    25 & 1 & 0 \\
    25 & 0 & 1 
    \end{pmatrix} 
    \)
    \item \( 
    \begin{pmatrix}
    1 & 0 & 0 \\
    24 & 1 & 0 \\
    24 & 0 & 1 
    \end{pmatrix} 
    \)
\end{enumerate}

\item If 
\[
y = \sum_{n=0}^{\infty} c_n x^n 
\]
is assumed to be a solution of the differential equation
\[
x^2 y'' - x y' - 3(1 + x^2) y = 0,
\]
then the values of \( r \) are

\begin{enumerate}
    \item 1 and 3
    \item -1 and 3
    \item 1 and -3
    \item -1 and -3
\end{enumerate}
\item Let the linear transformation \( T : \mathbb{F}^2 \rightarrow \mathbb{F}^3 \) be defined by 
\[ T(x_1, x_2) = (x_1, x_1 + x_2, x_2). \]
Then the nullity of \(T\) is:
\begin{enumerate}
    \item 0
    \item 1
    \item 2
    \item 3
\end{enumerate}

\item The approximate eigenvalue of the matrix
\[
A = \begin{bmatrix}
-15 & 4 & 3 \\
10 & -12 & 6 \\
20 & -4 & 2 
\end{bmatrix}
\]
obtained after two iterations of the Power method, with the initial vector \([1 \ 1 \ 1]^T\), is:
\begin{enumerate}
    \item 7.768
    \item 9.468
    \item 10.548
    \item 19.468
\end{enumerate}

\item The root of the equation \(xe^x = 1\) between 0 and 1, obtained by using two iterations of the bisection method, is:
\begin{enumerate}
    \item 0.25
    \item 0.50
    \item 0.75
    \item 0.65
\end{enumerate}
\textbf{Q.26 to Q.55 carry two marks each.}
\item  Let 
\[
\oint_C \left( \frac{1}{(z - 2)^2} + \frac{(a - 2)^2}{z} + 4 \right) \, dz = 4\pi i,
\]
where the closed curve \(C\) is the triangle having vertices at:
\[
i, \quad \frac{1}{\sqrt{2}}, \quad \text{and} \quad \frac{-1}{\sqrt{2}} \, i,
\]
the integral being taken in the anti-clockwise direction. Then one value of \(a\) is:
\begin{enumerate}
    \item \(1 + i\)
    \item \(2 + i\)
    \item \(3 + i\)
    \item \(4 + i\)
\end{enumerate}

\end{enumerate}
\end{document} 
