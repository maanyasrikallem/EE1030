\let\negmedspace\undefined
\let\negthickspace\undefined
\documentclass[journal]{IEEEtran}
\usepackage[a5paper, margin=10mm, onecolumn]{geometry}
%\usepackage{lmodern} % Ensure lmodern is loaded for pdflatex
\usepackage{tfrupee} % Include tfrupee package

\setlength{\headheight}{1cm} % Set the height of the header box
\setlength{\headsep}{0mm}     % Set the distance between the header box and the top of the text

\usepackage{gvv-book}
\usepackage{gvv}
\usepackage{cite}
\usepackage{amsmath,amssymb,amsfonts,amsthm}
\usepackage{algorithmic}
\usepackage{graphicx}
\usepackage{textcomp}
\usepackage{xcolor}
\usepackage{txfonts}
\usepackage{listings}
\usepackage{enumitem}
\usepackage{mathtools}
\usepackage{gensymb}
\usepackage{comment}
\usepackage[breaklinks=true]{hyperref}
\usepackage{tkz-euclide} 
\usepackage{listings}
% \usepackage{gvv}                                        
\def\inputGnumericTable{}                                 
\usepackage[latin1]{inputenc}                                
\usepackage{color}                                            
\usepackage{array}                                            
\usepackage{longtable}                                       
\usepackage{calc}                                             
\usepackage{multirow}                                         
\usepackage{hhline}                                           
\usepackage{ifthen}                                           
\usepackage{lscape}
\begin{document}

\bibliographystyle{IEEEtran}
\vspace{3cm}

\title{MA-$2012$-$14$ to $26$}
\author{AI24BTECH11017 - MAANYA SRI
}
% \maketitle
% \newpage
% \bigskip
{\let\newpage\relax\maketitle}

\renewcommand{\thefigure}{\theenumi}
\renewcommand{\thetable}{\theenumi}
\setlength{\intextsep}{10pt} % Space between text and floats


\numberwithin{equation}{enumi}
\numberwithin{figure}{enumi}
\renewcommand{\thetable}{\theenumi}

\begin{enumerate} 

	\item Let $R = \mathbb{Z} \times \mathbb{Z} \times \mathbb{Z}$ and $I = \mathbb{Z} \times \mathbb{Z} \times \cbrak{0}$.  
Then which of the following statements is correct?

\begin{enumerate}
    \item $I$ is a maximal ideal but not a prime ideal of $R$.
    \item $I$ is a prime ideal but not a maximal ideal of $R$.
    \item $I$ is both maximal ideal as well as a prime ideal of $R$.
    \item $I$ is neither a maximal ideal nor a prime ideal of $R$.
\end{enumerate}

\item The function $u \brak{r, \theta}$ satisfying the Laplace equation
\begin{align*}
\frac{\partial^2 u}{\partial r^2} + \frac{1}{r} \frac{\partial u}{\partial r} + \frac{1}{r^2} \frac{\partial^2 u}{\partial \theta^2} = 0, \quad e < r < e^2
\end{align*}
subject to the conditions $u \brak{e, \theta} = 1$, $u \brak{e^2, \theta} = 0$ is

\begin{enumerate}
    \item $\ln \brak{\frac{e}{r}}$
    \item $\ln \brak{\frac{e}{r^2}}$
    \item $\ln \brak{\frac{e^2}{r}}$
    \item $\sum_{n=1}^{\infty} \left( \frac{r - e^2}{e - e^2} \right) \sin n\theta$
\end{enumerate}

\item The functional
\begin{align*}
	\int_{0}^{1} \left( {y^{\prime}}^2 + \brak{y + 2y^{\prime}} y^{\prime \prime} + kxyy^{\prime} + y^2 \right) \, dx 
	\\ \quad y \brak{0} = 0, \, y \brak{1} = 1, \, y^\prime \brak{0} = 2, \, y^\prime \brak{1} = 3
\end{align*}
is path independent if $k$ equals

\begin{enumerate}
    \item 1
    \item 2
    \item 3
    \item 4
\end{enumerate}

\item If a transformation $y = u v$ transforms the given differential equation 
\begin{align*}
f \brak{x} y^{\prime\prime} - 4 f^\prime \brak{x} y^\prime + g \brak{x} y = 0 
\end{align*}
into the equation of the form $v^{\prime\prime} + h \brak{x} v = 0$, then $u$ must be

\begin{enumerate}
    \item $\frac{1}{f^2}$
    \item $x f$
    \item $\frac{1}{2 f}$
    \item $f^2$
\end{enumerate}

\item The expression 
\begin{align*}
\frac{1}{D_x^2 - D_y^2} \sin \brak{x - y}
\end{align*}
is equal to

\begin{enumerate}
    \item $-\frac{x}{2} \cos \brak{x - y}$
    \item $-\frac{x}{2} \sin \brak{x - y} + \cos \brak{x - y}$
    \item $\frac{x}{2} \cos \brak{x - y} + \sin \brak{x - y}$
    \item $\frac{3x}{2} \sin \brak{x - y}$
\end{enumerate}

\item The function $\phi \brak{x}$ satisfying the integral equation
\begin{align*}
\int_0^x e^{x-\xi} \phi \brak{\xi} \, d\xi = \frac{x^2}{2}
\end{align*}
is:

\begin{enumerate}
    \item $\frac{x^2}{2}$
    \item $x + \frac{x^2}{2}$
    \item $x - \frac{x^2}{2}$
    \item $1 + \frac{x^2}{2}$
\end{enumerate}

\item Given the data:

	\begin{table}[h!]
		\centering
		\begin{tabular}[12pt]{ |c| c|}
    \hline
    \textbf{Condition} & \textbf{Inference} \\
    \hline
     $\|x - C\|^2 < r^2$ & point lies inside the circle \\ 
     \hline
     $\|x - C\|^2 > r^2$ & point lies outside the circle \\ 
     \hline
     $\|x - C\|^2 = r^2$ & point lies on the circle \\ 
     \hline  
    \end{tabular}

	\end{table}

If the derivative of $y \brak{x}$ is approximated as:
\begin{align*}
y^\prime \brak{x_k} = \frac{1}{h} \left( \Delta y_k + \frac{1}{2} \Delta^2 y_k - \frac{1}{4} \Delta^3 y_k \right),
\end{align*}
then the value of $y^\prime \brak{2}$ is

\begin{enumerate}
    \item 4
    \item 8
    \item 12
    \item 16
\end{enumerate}

\item If 
\begin{align*}
A = \begin{pmatrix}
1 & 0 & 0 \\
1 & 0 & 1 \\
0 & 1 & 0
\end{pmatrix},
\end{align*}
then $A^{50}$ is

\begin{enumerate}
    \item $\begin{pmatrix} 1 & 0 & 0 \\ 50 & 1 & 0 \\ 50 & 0 & 1 \end{pmatrix}$
    \item $\begin{pmatrix} 1 & 0 & 0 \\ 48 & 1 & 0 \\ 48 & 0 & 1 \end{pmatrix}$
    \item $\begin{pmatrix} 1 & 0 & 0 \\ 25 & 1 & 0 \\ 25 & 0 & 1 \end{pmatrix}$
    \item $\begin{pmatrix} 1 & 0 & 0 \\ 24 & 1 & 0 \\ 24 & 0 & 1 \end{pmatrix}$
\end{enumerate}
\item If $y = \sum_{m=0}^{\infty} c_m x^{r+m}$ is assumed to be a solution of the differential equation

	\begin{align*}
		x^2 y^{\prime \prime} - xy^{\prime} - 3 \brak{1 + x^2} y = 0,
	\end{align*}

then the values of $r$ are
\begin{enumerate}
    \item $1$ and $3$
    \item $-1$ and $3$
    \item $1$ and $-3$
    \item $-1$ and $-3$
\end{enumerate}
\item Let the linear transformation $T : F^2 \rightarrow F^3$ be defined by
\begin{align*}
T \brak{x_1, x_2} = \brak{x_1, x_1 + x_2, x_2}.
\end{align*}
Then the nullity of $T$ is

\begin{enumerate}
    \item 0
    \item 1
    \item 2
    \item 3
\end{enumerate}

\item The approximate eigenvalue of the matrix
\begin{align*}
A = \begin{bmatrix}
-15 & 4 & 3 \\
10 & -12 & 6 \\
20 & -4 & 2 
\end{bmatrix}
\end{align*}
obtained after two iterations of Power method, with the initial vector $\sbrak{1 \ 1 \ 1}^T$ is

\begin{enumerate}
    \item 7.768
    \item 9.468
    \item 10.548
    \item 19.468
\end{enumerate}
\item The root of the equation $x e^x = 1$ between 0 and 1, obtained by using two iterations of bisection method, is
\begin{enumerate}
    \item 0.25
    \item 0.50
    \item 0.75
    \item 0.65
\end{enumerate}

\textbf{ Q. \ref{q26} to Q. 55 carry two marks each.}

\item \label{q26} Let 
\begin{align*}
	\int_C \sbrak{ \frac{1}{\brak{z - 2}^2} + \frac{\brak{a - 2}^2}{z} + 4} dz = 4\pi ,
\end{align*}
where the close curve $C$ is the triangle having vertices at
\begin{align*}
	i, \brak{ \frac{-1-i}{\sqrt{2}}}, \quad \text{and} \quad \brak{\frac{1-i}{\sqrt{2}}}
\end{align*}
the integral being taken in the anti-clockwise direction. Then one value of $a$ is
\begin{enumerate}
    \item $1 + i$
    \item $2 + i$
    \item $3 + i$
    \item $4 + i$
\end{enumerate}

 

\end{enumerate}

\end{document}

