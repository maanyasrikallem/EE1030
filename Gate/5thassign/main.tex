\let\negmedspace\undefined
\let\negthickspace\undefined
\documentclass[journal]{IEEEtran}
\usepackage[a5paper, margin=10mm, onecolumn]{geometry}
%\usepackage{lmodern} % Ensure lmodern is loaded for pdflatex
\usepackage{tfrupee} % Include tfrupee package

\setlength{\headheight}{1cm} % Set the height of the header box
\setlength{\headsep}{0mm}     % Set the distance between the header box and the top of the text

\usepackage{gvv-book}
\usepackage{gvv}
\usepackage{cite}
\usepackage{amsmath,amssymb,amsfonts,amsthm}
\usepackage{algorithmic}
\usepackage{graphicx}
\usepackage{textcomp}
\usepackage{xcolor}
\usepackage{txfonts}
\usepackage{listings}
\usepackage{enumitem}
\usepackage{mathtools}
\usepackage{gensymb}
\usepackage{comment}
\usepackage[breaklinks=true]{hyperref}
\usepackage{tkz-euclide} 
\usepackage{listings}
% \usepackage{gvv}                                        
\def\inputGnumericTable{}                                 
\usepackage[latin1]{inputenc}                                
\usepackage{color}                                            
\usepackage{array}                                            
\usepackage{longtable}                                       
\usepackage{calc}                                             
\usepackage{multirow}                                         
\usepackage{hhline}                                           
\usepackage{ifthen}                                           
\usepackage{lscape}
\begin{document}

\bibliographystyle{IEEEtran}
\vspace{3cm}

\title{ST-$2021$-$14$ to $26$}
\author{AI24BTECH11017 - MAANYA SRI
}
% \maketitle
% \newpage
% \bigskip
{\let\newpage\relax\maketitle}

\renewcommand{\thefigure}{\theenumi}
\renewcommand{\thetable}{\theenumi}
\setlength{\intextsep}{10pt} % Space between text and floats


\numberwithin{equation}{enumi}
\numberwithin{figure}{enumi}
\renewcommand{\thetable}{\theenumi}

\begin{enumerate} 
\item Let $A$ and $B$ be two events such that $P\brak{B} = \frac{3}{4}$ and $P\brak{A \cup B^c} = \frac{1}{2}$. If $A$ and $B$ are independent, then $P\brak{A}$ equals \underline{\hspace{1cm}} (round off to 2 decimal places).

\item A fair die is rolled twice independently. Let $X$ and $Y$ denote the outcomes of the first and second roll, respectively. Then $E\brak{X+Y \mid \brak{X - Y}^2 = 1}$ equals \underline{\hspace{1cm}}.

\item Let $X$ be a random variable having distribution function
\begin{align*}
F\brak{x} = 
\begin{cases}
0, & x < 1, \\
\frac{a}{2}, & 1 \leq x < 2, \\
\frac{c}{6}, & 2 \leq x < 3, \\
1, & x \geq 3,
\end{cases}
\end{align*}
where $a$ and $c$ are appropriate constants. Let $A_n = \sbrak{ 1 + \frac{1}{n}, 3 - \frac{1}{n} }, n \geq 1$, and $A = \bigcup_{i=1}^{\infty} A_i$.
If $P\brak{X \leq 1} = \frac{1}{2}$ and $E\brak{X} = \frac{5}{3}$, then $P\brak{X \in A}$ equals \underline{\hspace{1cm}} (round off to 2 decimal places).

\item If the marginal probability density function of the $k^{th}$ order statistic of a random sample of size 8 from a uniform distribution on $\sbrak{0,2}$ is
\begin{align*}
f\brak{x} = 
\begin{cases}
\frac{7}{32} x^6 \brak{2 - x}, & 0 < x < 2, \\
0, & \text{otherwise},
\end{cases}
\end{align*}
then $k$ equals \underline{\hspace{1cm}}.

\item For $\alpha > 0$, let $\cbrak{ X_n^{\brak{\alpha}} }_{n\geq1}$ be a sequence of independent random variables such that
\begin{align*}
P\brak{X_n^{\brak{\alpha}} = 1} = \frac{1}{n^{2\alpha}} = \quad 1 - P\brak{X_n^{\brak{\alpha}} = 0}.
\end{align*}
Let $S = \cbrak{ \alpha > 0 : X_n^{\brak{\alpha}} \text{ converges to 0 almost surely as } n \to \infty }$.
Then the infimum of $S$ equals \underline{\hspace{1cm}} (round off to 2 decimal places).

\item Let $\cbrak{X_n}_{n\geq1}$ be a sequence of independent and identically distributed random variables each having uniform distribution on $\sbrak{0,2}$. For $n \geq 1$, let
\begin{align*}
Z_n = - \log_e \brak{ \prod_{i=1}^{n} \brak{2 - X_i} }^{\frac{1}{n}}.
\end{align*}
Then, as $n \to \infty$, the sequence $\cbrak{Z_n}_{n\geq1}$ converges almost surely to \underline{\hspace{1cm}} (round off to 2 decimal places).

\item Let $\cbrak{X_n}_{n \geq 0}$ be a time-homogeneous discrete time Markov chain with state space $\cbrak{0,1}$ and transition probability matrix
\begin{align*}
\begin{bmatrix}
0.25 & 0.75 \\
0.75 & 0.25 
\end{bmatrix}.
\end{align*}
If $P\brak{X_0 = 0} = P\brak{X_0 = 1} = 0.5$, then
\begin{align*}
\sum_{k=1}^{100} E \sbrak{\brak{X_{2k}}^{2k}}
\end{align*}
equals \underline{\hspace{1cm}}

\item Let $\cbrak{0,2}$ be a realization of a random sample of size 2 from a binomial distribution with parameters 2 and $p$, where $p \in \brak{0,1}$. To test $H_0: p = \frac{1}{2}$ against $H_1: p \neq \frac{1}{2}$, the observed value of the likelihood ratio test statistic equals \underline{\hspace{1cm}} (round off to 2 decimal places).

\item Let $X$ be a random variable having the probability density function
\begin{align*}
f\brak{x} = 
\begin{cases}
\frac{3}{13} \brak{1 - x} \brak{9 - x}, & 0 < x < 1, \\
0, & \text{otherwise}.
\end{cases}
\end{align*}
Then $\frac{4}{3} E \sbrak{X \brak{X^2 - 15X + 27}}$ equals \underline{\hspace{1cm}} (round off to 2 decimal places).

\item Let $\brak{Y, X_1, X_2}$ be a random vector with mean vector $\begin{pmatrix} 5 \\ 2 \\ 0 \end{pmatrix}$ and variance-covariance matrix
\begin{align*}
\begin{bmatrix}
10 & 0.5 & -0.5 \\
0.5 & 7 & 1.5 \\
-0.5 & 1.5 & 2
\end{bmatrix}.
\end{align*}
Then the value of the multiple correlation coefficient between $Y$ and its best linear predictor on $X_1$ and $X_2$ equals \underline{\hspace{1cm}} (round off to 2 decimal places).

\item Let $X_1, X_2$ and $X_3$ be a random sample from a bivariate normal distribution with unknown mean vector $\mu$ and unknown variance-covariance matrix $\Sigma$, which is a positive definite matrix. The $p$-value corresponding to the likelihood ratio test for testing $H_0: \mu = 0$ against $H_1: \mu \neq 0$ based on the realization $\cbrak{ \begin{pmatrix} 1 \\ 2 \end{pmatrix}, \begin{pmatrix} -4 \\ -2 \end{pmatrix}, \begin{pmatrix} -5 \\ 0 \end{pmatrix}}$  of the random sample equals \underline{\hspace{1cm}} (round off to 2 decimal places).

\item Let $Y_i = \alpha + \beta x_i + \epsilon_i$, $i = 1, 2, 3$, where $x_i$'s are fixed covariates, $\alpha$ and $\beta$ are unknown parameters and $\epsilon_i$'s are independent and identically distributed random variables with mean zero and finite variance. Let $\hat{\alpha}$ and $\hat{\beta}$ be the ordinary least squares estimators of $\alpha$ and $\beta$, respectively. Given the following observations
 \begin{table}[h!]
     \centering
     \begin{tabular}[12pt]{ |c| c|}
    \hline
    \textbf{Condition} & \textbf{Inference} \\
    \hline
     $\|x - C\|^2 < r^2$ & point lies inside the circle \\ 
     \hline
     $\|x - C\|^2 > r^2$ & point lies outside the circle \\ 
     \hline
     $\|x - C\|^2 = r^2$ & point lies on the circle \\ 
     \hline  
    \end{tabular}

 \end{table}
the value of $\hat{\alpha} + \hat{\beta}$ equals \underline{\hspace{1cm}} (round off to 2 decimal places).
\\ \textbf{ Q. \ref{q26}- Q. 43 Multiple Choice Question (MCQ), carry TWO mark each(for each wrong answer : -2/3).}
\item \label{q26} Let $f: \mathbb{R} \rightarrow \mathbb{R}$ be defined by
\begin{align*}
f\brak{x} =
\begin{cases}
x^3 \sin x, & x = 0 \text{ or } x \text{ is irrational}, \\
\frac{1}{q^3}, & x = \frac{p}{q}, p \in \mathbb{Z} \setminus \cbrak{0}, q \in \mathbb{N} \text{ and } \text{gcd}(p,q) = 1,
\end{cases}
\end{align*}
where $\mathbb{R}$ denotes the set of all real numbers, $\mathbb{Z}$ denotes the set of all integers, $\mathbb{N}$ denotes the set of all positive integers and $\text{gcd}\brak{p, q}$ denotes the greatest common divisor of $p$ and $q$. Then which one of the following statements is true?

\begin{enumerate}
\item $f$ is not continuous at 0
\item $f$ is not differentiable at 0
\item $f$ is differentiable at 0 and the derivative of $f$ at 0 equals 0
\item $f$ is differentiable at 0 and the derivative of $f$ at zero equals 1
\end{enumerate}

 \end{enumerate}
 \end{document}
