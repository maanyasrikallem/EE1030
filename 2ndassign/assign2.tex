\let\negmedspace\undefined
\let\negthickspace\undefined
\documentclass[journal]{IEEEtran}
\usepackage[a5paper, margin=10mm, onecolumn]{geometry}
%\usepackage{lmodern} % Ensure lmodern is loaded for pdflatex
\usepackage{tfrupee} % Include tfrupee package

\setlength{\headheight}{1cm} % Set the height of the header box
\setlength{\headsep}{0mm}     % Set the distance between the header box and the top of the text

\usepackage{gvv-book}
\usepackage{gvv}
\usepackage{cite}
\usepackage{amsmath,amssymb,amsfonts,amsthm}
\usepackage{algorithmic}
\usepackage{graphicx}
\usepackage{textcomp}
\usepackage{xcolor}
\usepackage{txfonts}
\usepackage{listings}
\usepackage{enumitem}
\usepackage{mathtools}
\usepackage{gensymb}
\usepackage{comment}
\usepackage[breaklinks=true]{hyperref}
\usepackage{tkz-euclide} 
\usepackage{listings}
% \usepackage{gvv}                                        
\def\inputGnumericTable{}                                 
\usepackage[latin1]{inputenc}                                
\usepackage{color}                                            
\usepackage{array}                                            
\usepackage{longtable}                                       
\usepackage{calc}                                             
\usepackage{multirow}                                         
\usepackage{hhline}                                           
\usepackage{ifthen}                                           
\usepackage{lscape}
\begin{document}

\bibliographystyle{IEEEtran}
\vspace{3cm}

\title{1-1.3-2}
\author{AI24BTECH11017-Maanya Sri
}
% \maketitle
% \newpage
% \bigskip
{\let\newpage\relax\maketitle}

\renewcommand{\thefigure}{\theenumi}
\renewcommand{\thetable}{\theenumi}
\setlength{\intextsep}{10pt} % Space between text and floats


\numberwithin{equation}{enumi}
\numberwithin{figure}{enumi}
\renewcommand{\thetable}{\theenumi}
\textbf{Question}:\\
The coordinates of the three consecutive vertices of a parallelogram $ABCD$ are $\textbf{A} (1,3), \textbf{B}(-1,2)$, and $\textbf{C}(2,5)$.Find the coordinates of the fourth vertex \textbf{D}.
\hfill(10,2021)
\\
\textbf{Sol:}
Given $A$=\myvec{
1
\\
3
},
$B$=\myvec{
-1
\\
2
}, and
$C$=\myvec{
2
\\
5
}
\newline Let $A = 1i+3j B = -1i+2j C = 2i+5j$
In a parallelogram $ABCD, D$ can be written as $D=A+C-B
          = 1i+3j+2i+5j-(-1i+2j)
          = 4i+6j$
Hence $D$ = \myvec{
4
\\
6
}
\end{document}

