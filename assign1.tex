%iffalse
\let\negmedspace\undefined
\let\negthickspace\undefined
\documentclass[journal,12pt,twocolumn]{IEEEtran}
\usepackage{cite}
\usepackage{amsmath,amssymb,amsfonts,amsthm}
\usepackage{algorithmic}
\usepackage{graphicx}
\usepackage{textcomp}
\usepackage{xcolor}
\usepackage{txfonts}
\usepackage{listings}
\usepackage{enumitem}
\usepackage{mathtools}
\usepackage{gensymb}
\usepackage{comment}
\usepackage[breaklinks=true]{hyperref}
\usepackage{tkz-euclide} 
\usepackage{listings}
\usepackage{gvv}                                        
%\def\inputGnumericTable{}                                 
\usepackage[latin1]{inputenc}                                
\usepackage{color}                                            
\usepackage{array}                                            
\usepackage{longtable}                                       
\usepackage{calc}                                             
\usepackage{multirow}                                         
\usepackage{hhline}                                           
\usepackage{ifthen}                                           
\usepackage{lscape}
\usepackage{tabularx}
\usepackage{array}
\usepackage{float}


\newtheorem{theorem}{Theorem}[section]
\newtheorem{problem}{Problem}
\newtheorem{proposition}{Proposition}[section]
\newtheorem{lemma}{Lemma}[section]
\newtheorem{corollary}[theorem]{Corollary}\newtheorem{example}{Example}[section]
\newtheorem{definition}[problem]{Definition}
\newcommand{\BEQA}{\begin{eqnarray}}
\newcommand{\EEQA}{\end{eqnarray}}
\newcommand{\define}{\stackrel{\triangle}{=}}
\theoremstyle{remark}
\newtheorem{rem}{Remark}

% Marks the beginning of the document
\begin{document}
\bibliographystyle{IEEEtran}
\vspace{3cm}

\title{Matrix Theory 1st Assignment}
\author{AI24BTECH11017 - Maanya sri}
\maketitle
\newpage
\bigskip

\renewcommand{\thefigure}{\theenumi}
\renewcommand{\thetable}{\theenumi}

\begin{enumerate}[start=16] 
\item Using mathematical induction, prove that $\tan^{-1}\brak{1/3}  +  \tan^{-1}\brak{1/7} + .....\tan^{-1}\{1/\brak{n^2 + n + 1}\} = \tan^{-1}\{\brak{n/n+2}\}$ 

$\hfill\textcolor{black}{\brak{1993-5 Marks} }$
\item Prove that $\sum\limits^{k}_{r=1} \brak{-3}^{r-1}$ ${}^{3n}C_{2r-1}$ = $0$, where $k=\brak{3n}/2$ and $n$ is an even positive integer.
$\hfill\textcolor{black}{\brak{1993-5 Marks} }$
\item If $x$ is not an integral multiple of $2\pi$ use mathematical induction to prove that : 
cos $x$ + cos $2x$ + ........ + cos $nx$ = cos$\frac{n+1}{2}$ $x$ sin $\frac{nx}{2}$ cosec$\frac{x}{2}$
$\hfill\textcolor{black}{\brak{1994-4 Marks} }$
\item 
Let $n$ be a positive integer and $\brak{1+x+x^2}^n$ = $a_0 + a_1 x + ............+ a_{2n} x^{2n}$. Show that $a_0^2 - a_1^2 + a_2^2 ............ + a_2n^2 = a_n$
$\hfill\textcolor{black}{\brak{1994-5 Marks} }$
\item 
Using mathematical induction prove that for every integer $n \geq 1$, $\brak{3^{2n} -1}$ is divisible by $2^{n+2}$ but not by $2^{n+3}$.
$\hfill\textcolor{black}{\brak{1996-3 Marks} }$
\item 
Let $ 0<A_i<\pi$ for $i= 1,2....,n$. Use mathematical induction to prove that sin $A_1$ + sin $A_2$... + sin $A_n \leq  n$  sin $\brak{\frac{A_1 + A_2 +.....+ A_n}{n}}$ where $\geq$ 1 is a natural number .\sbrak{ You may use the fact that $p$ sin $x$ + $\brak{1-p}$ sin $y\leq$ sin$\sbrak{px + \brak{1-p}y}$, where $0 \leq p \leq 1 $ and 0 $\leq x,y \leq \pi }$.
$\hfill\textcolor{black}{\brak{1997-5 Marks} }$\item 
Let $p$ be a prime number and $m$ a positive integer. By mathematical induction on $m$, or otherwise, prove that whenever $r$ is an integer such that $p$ does not divide $r$, $p$ divides ${}^{mp}C_r$ 
		$\lsbrak{\textbf{Hint:}}$ You may use the fact that $\brak{1+x}^{\brak{m+1}p} = \brak{1+x}^p\brak{1+x}^ {mp} \rbrak{}$
$\hfill\textcolor{black}{\brak{1998-8 Marks} }$
\item 
Let $n$ be any positive integer. Prove that $\sum\limits^{m}_{k=0}\frac{\binom{2n-k}{k}}{\binom{2n-1}{n}}.\frac{2n-4k+1}{2n-2k+1}2^{n-2k}= \frac{\binom{n}{m}}{\binom{2n-2m}{n-m} } 2^{n-2m}$for each non-be gatuve integer $m\leq n. \brak{Here \binom{p}{q} = {}^pC_q}.$
$\hfill\textcolor{black}{\brak{1999-10 Marks} }$
\item
For any positive integer $m,n\brak{with n \geq m}$, let $\binom{n}{m} ={}^nC_m$ . Prove that $\binom{n}{m} + \binom{n-1}{m} + \binom{n-2}{m} + ..... + \binom{m}{m} = \binom{n+1}{m+2}$ . Hence or otherwise, prove that $\binom{n}{m} +2\binom{n-1}{m} +3\binom{n-2}{m}+....+ \brak{n-m+1} \binom{m}{m} = \binom{n+2}{m+2}$.
$\hfill\textcolor{black}{\brak{2000-6 Marks} }$
\item
For every positive integer n, prove that $\sqrt{\brak{4n+1}} < \sqrt{n} +\sqrt{n+1} < \sqrt{4n+2}$. Hence or otherwise, prove that $\sbrak{\sqrt{n} + \sqrt{\brak{n+1}}}= \sbrak{\sqrt{4n+1}},$ where $\sbrak{x}$ denotes the greatest integer not exceeding $x$.
$\hfill\textcolor{black}{\brak{2000-6 Marks} }$
\item
Let $a,b,c$ be positive real numbers such that $b^2 - 4ac > 0$ and let $\alpha_1 = c$. Prove by induction that $\alpha_{n+1} = \frac{a \alpha_{n^2}}{\brak{b^2 - 2a \brak{\alpha_1 + \alpha_2 + ... \alpha_n}}}$ is well-defined and $\alpha_{n+1} < \frac{\alpha_n}{2}$ for all $n = 1,2,... \brak{Here,'well-defined' means that the denominator in the expression for \alpha_n+1 is not zero.}$
$\hfill\textcolor{black}{\brak{2001-5 Marks} }$
\item
Use mathematical induction to show that $\brak{25}^{n+1}$ - 24 $n$ + 5735 is divisible by $\brak{24}^2$ for all $n$= 1,2,.....
$\hfill\textcolor{black}{\brak{2002-5 Marks} }$
\item 
Prove that $2^k \binom{n}{0}\binom{n}{k} - 2^{k-1\binom{n}{2}} \binom{n}{1} \binom{n-1}{k-1} + 2^{k-2}\binom{n-2}{k-2} - .....\brak{-1}^k \binom{n}{k}\binom{n-k}{0} = \binom{n}{k}$.
$\hfill\textcolor{black}{\brak{2003-2 Marks} }$
\item
A coin has probability p of showing head when tossed. It is tossed n times.Let $p_n$ denote the probability that no two (or more) consequtive heads occur. Prove that $p_1=1$ $p_2=1-p^2$ and $p_n$ = $(1-p).p_{n-1} + p(1-p)p_{n-2}$ for all n $\geq$ 3. Prove by induction on $n$, that $p_n$ = $A\alpha^n + B\beta^n$ for all $n$ $\geq$ 1, where $\alpha$ and $\beta$ are the roots of quadratic equation $x^2$ - $\brak{1-p}x - p\brak{1-p}$ = 0 and A = $\frac{p^2 + \beta -1}{\alpha\beta-\alpha^2}$, B = $\frac{p^2 + \alpha -1}{\alpha\beta - \beta^2}$.
$\hfill\textcolor{black}{\brak{2000-5 Marks} }$



\end{enumerate}
\end{document}

