\let\negmedspace\undefined
\let\negthickspace\undefined
\documentclass[article,12pt,onecolumn]{IEEEtran}
\usepackage{cite}
\usepackage{amsmath,amssymb,amsfonts,amsthm}
\usepackage{algorithmic}
\usepackage{graphicx}
\usepackage{textcomp}
\usepackage{xcolor}
\usepackage{txfonts}
\usepackage{listings}
\usepackage{enumitem}
\usepackage{mathtools}
\usepackage{gensymb}
\usepackage{comment}
\usepackage[breaklinks=true]{hyperref}
\usepackage{tkz-euclide} 
\usepackage{listings}
\usepackage{gvv}                                        
%\def\inputGnumericTable{}\usepackage[latin1]{inputenc}                                
\usepackage{color}                                            
\usepackage{array}                                            
\usepackage{longtable}                                       
\usepackage{calc}                                             
\usepackage{multirow}                                         
\usepackage{hhline}                                           
\usepackage{ifthen}                                           
\usepackage{lscape}
\usepackage{tabularx}
\usepackage{array}
\usepackage{float}


\newtheorem{theorem}{Theorem}[section]
\newtheorem{problem}{Problem}
\newtheorem{proposition}{Proposition}[section]
\newtheorem{lemma}{Lemma}[section]
\newtheorem{corollary}[theorem]{Corollary}
\newtheorem{example}{Example}[section]
\newtheorem{definition}[problem]{Definition}
\newcommand{\BEQA}{\begin{eqnarray}}
\newcommand{\EEQA}{\end{eqnarray}}\newcommand{\define}{\stackrel{\triangle}{=}}
\theoremstyle{remark}
\newtheorem{rem}{Remark}

% Marks the beginning of the document
\begin{document}
\bibliographystyle{IEEEtran}
\vspace{3cm}

\title{Matrix Theory 1st Assignment}
\author{AI24BTECH11017 - Maanya sri}
\maketitle
\newpage
\bigskip

\renewcommand{\thefigure}{\theenumi}
\renewcommand{\thetable}{\theenumi}
\begin{enumerate}[start=16] 
\item Using mathematical induction, prove that $\tan^{-1}\brak{\frac{1}{3}}  +  \tan^{-1}\brak{\frac{1}{7}} + .....\tan^{-1}\cbrak{\frac{1}{\brak{n^2 + n + 1}}} = \tan^{-1}\cbrak{\frac{n}{\brak{n+2}}}$ 
\hfill(1993-5 Marks) 
\item Prove that $\sum\limits^{k}_{r=1} \brak{-3}^{r-1} \comb{3n}{2n-1} = 0$, where $k=\frac{(3n)}{2}$ and $n$ is an even positive integer.
\hfill(1993-5 Marks)
\item If $x$ is not an integral multiple of $2\pi$ use mathematical induction to prove that : 
$\cos{x} + \cos{2x} + ........ + \cos{nx} = \cos{\frac{n+1}{2}x}\sin{\frac{nx}{2}}\cosec{\frac{x}{2}}$
\hfill(1994-4 Marks)
\item 
Let $n$ be a positive integer and $\brak{1+x+x^2}^n = a_0 + a_1 x + ............+ a_{2n} x^{2n}$. Show that $a_0^2 - a_1^2 + a_2^2 ............ + a_2n^2 = a_n$
\hfill(1994-5 Marks)
\item 
Using mathematical induction prove that for every integer $n \geq 1 , \brak{3^{2n} -1}$ is divisible by $2^{n+2}$ but not by $2^{n+3}$.
\hfill(1996-3 Marks)
\item 
Let $0<A_i<\pi$ for $i= 1,2....,n$. Use mathematical induction to prove that $\sin{A_1} + \sin{A_2}... + \sin{A_n} \leq  n \sin{\brak{\frac{A_1 + A_2 +.....+ A_n}{n}}}$ where $\geq 1$ is a natural number .\{You may use the fact that $p \sin{x} + \brak{1-p}\sin{y}\leq \sin{\sbrak{px + \brak{1-p}y}}$, where $0 \leq p \leq 1$ and $0 \leq x,y \leq \pi.\}$
\hfill(1997-5 Marks)
\item 
Let $p$ be a prime number and $m$ a positive integer. By mathematical induction on $m$, or otherwise, prove that whenever $r$ is an integer such that $p$ does not divide $r , p$ divides $\comb{mp}{r}$
$[\textbf{Hint:}$ You may use the fact that $\brak{1+x}^{\brak{m+1}p} = \brak{1+x}^p\brak{1+x}^{mp}]$
\hfill(1998-8 Marks)
\item 
Let $n$ be any positive integer. Prove that $\sum\limits^{m}_{k=0}\frac{\binom{2n-k}{k}}{\binom{2n-1}{n}}.\frac{2n-4k+1}{2n-2k+1}2^{n-2k}= \frac{\binom{n}{m}}{\binom{2n-2m}{n-m}} 2^{n-2m}$for each non-be gatuve integer $m\leq n$. (Here $\binom{p}{q} = \comb{p}{q}$).
\hfill(1999-10 Marks)
\item
For any positive integer $m,n$(with $n \geq m$), let $\binom{n}{m} =\comb{n}{m}$ . Prove that $\binom{n}{m} + \binom{n-1}{m} + \binom{n-2}{m} + ..... + \binom{m}{m} = \binom{n+1}{m+2}$ . Hence or otherwise, prove that 
\begin{align}
\binom{n}{m} +2\binom{n-1}{m} +3\binom{n-2}{m}+....+ \brak{n-m+1} \binom{m}{m} = \binom{n+2}{m+2}
\end{align}.
\hfill(2000-6 Marks)
\item
For every positive integer $n$, prove that $\sqrt{(4n+1)} < \sqrt{n} +\sqrt{n+1} < \sqrt{4n+2}$. Hence or otherwise, prove that $\sbrak{\sqrt{n} + \sqrt{\brak{n+1}}}= \sbrak{\sqrt{4n+1}},$ where $\sbrak{x}$ denotes the greatest integer not exceeding $x$.
\hfill(2000-6 Marks)
\item
Let $a,b,c$ be positive real numbers such that $b^2 - 4ac > 0$ and let $\alpha_1 = c$. Prove by induction that $\alpha_{n+1} = \frac{a\alpha_{n}^2}{\brak{b^2 - 2a \brak{\alpha_1 + \alpha_2 + ... \alpha_n}}}$ is well-defined and $\alpha_{n+1} < \frac{\alpha_n}{2}$ for all $n$ = 1,2,...
(Here,'well-defined' means that the denominator in the expression for $\alpha_{n+1}$ is not zero.)
\hfill(2001-5 Marks)
\item
Use mathematical induction to show that $\brak{25}^{n+1} - 24 n + 5735$ is divisible by $\brak{24}^2$ for all $n$= 1,2,.....
\hfill(2002-5 Marks)
\item 
Prove that
\begin{align}
2^k \binom{n}{0}\binom{n}{k} - 2^{k-1\binom{n}{2}} \binom{n}{1} \binom{n-1}{k-1} + 2^{k-2}\binom{n-2}{k-2} - .....\brak{-1}^k \binom{n}{k}\binom{n-k}{0} = \binom{n}{k}
\end{align}.
\hfill(2003-2 Marks)
\item
A coin has probability $p$ of showing head when tossed. It is tossed $n$ times.Let $p_n$ denote the probability that no two (or more) consequtive heads occur. Prove that $p_1=1 , p_2=1-p^2$ and $p_n = (1-p).p_{n-1} + p(1-p)p_{n-2}$ for all $ n \geq 3 $. Prove by induction on $n$, that $p_n$ = $A\alpha^n + B\beta^n$ for all $n \geq 1$, where $\alpha$ and $\beta$ are the roots of quadratic equation $x^2 - \brak{1-p}x - p\brak{1-p} = 0$ and A = $\frac{p^2 + \beta -1}{\alpha\beta-\alpha^2}$, B = $\frac{p^2 + \alpha -1}{\alpha\beta - \beta^2}$.
\hfill(2000-5 Marks)



\end{enumerate}
\end{document}


